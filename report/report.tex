\documentclass{article}
\usepackage{graphicx} % Required for inserting images
\usepackage{cite}
\usepackage{setspace}

\title{NHL Meter}
\author{Alexander Hagood \& Alexander Barran}
\date{November 2024}
\doublespacing
\begin{document}

\maketitle

\section{Abstract}
The objective of this project is to provide a system that can provide live estimations
about the outcomes of a National Hockey League (NHL) game. Using statistical and
machine learning methods, we were able to create a model that can ???
\section{Introduction}
Inspired by visualizations like the Win Probability graph provided by ESPN, we sought
out to create a similar metric based on Hockey play by play data sourced from the NHL.
We would like to use information about the state of the game, such as the current 
score, 'momentum,' and other factors that can be derived from the data to determine 
the odds of the home team winning at any arbitrary point in the game. Sports are 
inherently unpredictable, however we can still make educated guesses at the chances 
of specific outcomes based on information we have. Our approach is to stream in ‘Play-By-Play’ 
data representing certain game events into our model, and then use statistical and machine 
learning techniques like ??? to update our estimation on who is more likely to win as we 
gain more information about the flow of the game.


\section{Problem Definition}
Given the Play-By-Play data for a specific NHL match up to a certain time point, 
what is the probability of the Home or Away team winning? With the rise of the sports 
betting industry, these types of endeavors have grown in popularity and interest on 
both the bettor and bookkeepers sides, as bettors seek new edges and bookkeepers seek 
to set more accurate lines. Estimations of NHL outcomes are not new, but from our research 
there is no investigation into the dynamic estimation as the game is played. Measuring
this probability can give us insight into the nature of how an underdog team might
beat a strong team, or show how dominant a strong team can control the game.


\section{Models, Algorithms and Measures}

\section{Implementation and Analysis}
Our data is sourced from the official NHL API. This allows you to get historical play-by-play 
data from games since the 2007 season using HTTP requests. Using a Game ID, you can download 
all the games for a specific season. Thankfully, there exists an existing python implementation 
that scrapes this data. 'hockey-scraper' by harryshomer on pypi allows you to input a season
or a date range, and will download the season information then all associated plays with
each game in that season. This is a somewhat large amount of data, taking multiple days to 
download due to API rate limiting. Post-parquet compression, the data including play-by-play
and shifts (documenting when a player comes on or off the ice) took up approximately half a gigabyte.

Analyzing the format of the data, we can view useful information about the game. A "play" is 
recorded whenever a notable event occurs. These events can take forms like a shot on goal, 
a successful goal, the result of a face off, penalties, or other associated game events. 
For every play, all players that were involved are listed, as well as all other players on 
the ice at the time. Spatial data is included, such as the X and Y coordinates on the rink 
where the event happened, and of course information about the game itself such as the current 
score, the teams, and the time remaining in the period.

In order to start adjusting the probability of a given team winning, you need a beginning assumption.
You could just assume that each time has an equal chance of winning, but this naive assumption 
neglects variables such as the strength of a team. In order to get a rough estimate of likelihood 
to win, we can use a system like Elo. Elo is a metric that lets us measure the comparative strength 
of teams, derived from their record so far in the season. When you win, you get Elo points scaled 
to the differential between you and your opponents Elo score, and you lose points upon a loss. Using the formula 
\[E_{A} = \frac{1}{1 + 10^{(R_{B} - R_{A})/400}}\] we can calculate the percentage chance of a given team beating another.
This is a good starting point for our estimation, and now we can adjust this probability
as the game goes on to get a more accurate estimate of the outcome. 

\section{Related Work}
Similar work in this field has been done for other sports, such as the aforementioned ESPN Win Probability graph.
Some of the most useful work in this field has been done by groups studying Soccer like American Soccery Analysis using a 
machine learning based approach \cite{richardett} or research by Robberechts et al \cite{bayesian} using bayesian methods. Soccer
is a much more translatable sport to our problem, as it is also a low scoring game with a similar gameplay flow to hockey.
These works served as useful inspiration and insight into how we could develop an approach to predict the outcomes of hockey games.

\nocite{*}
\bibliographystyle{IEEEtran}
\bibliography{references}
\end{document}